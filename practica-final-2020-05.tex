%%----------------------------------------------------------------------------
%%----------------------------------------------------------------------------
\section{Práctica final: MisCosas (2020, mayo)}
\label{practica-final-2020-05}

[ \textbf{Nota importante:} Por ahora esto es sólo es un borrador. Aún estamos definiendo cómo será el enunciado definitivo. ]
%[ \textbf{Nota importante:} Este enunciado es aún tentativo, y puede sufrir cambios ]

La práctica final de la asignatura consiste en la creación de una aplicación web, llamada ``MisCosas'', que permitirá gestionar vídeos, noticias y otra información que los usuarios vayan encontrando por la red y les resulte interesante. Los usuarios podrán ver los contenidos de sitios preseleccionados, añadir otros, elegir los que más les interesen, y compartir los que han seleccionado de distintas maneras. A continuación se describe el funcionamiento y la arquitectura general de la aplicación, la funcionalidad mínima que debe proporcionar, y otra funcionalidad optativa que podrá tener.

Por un lado, la aplicación se encargará de descargar información de varios sitios de Internet para permitir a los usuarios que puedan elegirla. Por otro, permitirá a los usuarios elegir, entre ellos, qué información quieren que se les muestre para realizar su selección, y podrán compartir estas selecciones con otros usuarios.

%%----------------------------------------------------------------------------
\subsection{Arquitectura y funcionamiento general}

Arquitectura general:

\begin{itemize}

\item La práctica se construirá como un proyecto Django/Python3, que incluirá una o varias aplicaciones (apps) Django que implementen la funcionalidad requerida.

\item Para el almacenamiento de datos persistente se usará SQLite3, con tablas definidas en modelos de Django.

\item Para implementar usuarios, cuando sea necesario, se usará como base el sistema de autenticación de usuarios que proporciona Django\footnote{User Authentication in Django:\\ \url{https://docs.djangoproject.com/en/3.0/topics/auth/}}.

\item Todas las bases de datos que contenga la aplicación tendrán que ser accesibles vía la interfaz que proporciona el ``Admin Site'' (además de lo que pueda hacer falta para que funcione al aplicación).

\item Se utilizarán plantillas Django (a ser posible, una jerarquía de plantillas, para que la práctica tenga un aspecto similar) para definir las páginas que se servirán a los navegadores de los usuarios. Estas plantillas incluirán en todas las páginas al menos:

  \begin{itemize}
  \item Un \emph{banner} (imagen) del sitio, preferentemente en la parte superior izquierda.
  \item Una caja para entrar (hacer login en el sitio), o para salir (si ya se ha entrado).
    \begin{itemize}
    \item En caso de que no se haya entrado en una cuenta, esta caja permitirá al visitante introducir su identificador de usuario y su contraseña, o crearse una cuenta.
    \item En caso de que ya se haya entrado, esta caja mostrará el identificador del usuario y permitirá salir de la cuenta (logout). Esta caja aparecerá preferentemente en la parte superior derecha.
    \end{itemize}
  \item Un menú de opciones, como barra, preferentemente debajo de los dos elementos anteriores (banner y caja de entrada o salida).
  \item Un pie de página con una nota de atribución, indicando ``Esta aplicación utiliza datos proporcionados por XXX, YYY y ZZZ'', siendo XXX, YYY y ZZZ los sitios desde donde se descarga información, y siendo cada uno de ellos un enlace al sitio en cuestión.
  \end{itemize}

  Cada una de estas partes estará construida dentro de un elemento ``div'', marcada con un atributo ``id'' en HTML, para poder ser referenciadas fácilmente en hojas de estilo CSS.

\item Se utilizarán hojas de estilo CSS para determinar la apariencia de la práctica. Estas hojas definirán al menos el color y el tamaño de la letra, y el color de fondo de cada una de las partes (elementos) marcadas con un \emph{id}, tal como se indica en el apartado anterior. Además, elementos que deban tener el mismo aspecto deberían estar en una misma clase, para poder gestionarlo de forma común.
  
\item Para obtener información de cada sitio de Internet soportado por la aplicación, se utilizará la API de ese sitio, o quizás en algunos casos, ser hará un análisis de las páginas HTML del sitio. En general, la forma de funcionamiento será la siguiente:

  \begin{itemize}
  \item Llamaremos ``alimentador'' a cada una de las fuentes de datos del sitio. Por ejemplo, en YouTube, cada alimentador será un canal.
  \item Llamaremos ``item'' a cada uno de los elementos de un alimentador. Por ejemplo, en YouTube, cada item será un vídeo.
  \item Se ofrecerá un elemento HTML que permita al usuario elegir qué alimentador se va a obtener del sitio.
  \item La información obtenida de ese alimentador se organizará como una lista de items, que se almacenará en la base de datos.
  \item A partir de lo almacenado en la base de datos, se ofrecerá al usuario la lista de items para la selección.
  \item Se ofrecerá una forma para actualizar la información.
  \end{itemize}

  Puede verse más información sobre los alimentadores en la sección~\ref{practica-final-2020-05:alimentadores}.
\end{itemize}

Funcionamiento general:

\begin{itemize}
\item En general, para utilizar el sitio, no hará falta autenticarse con una cuenta. Toda la funcionalidad estará disponible para cualquier visitante, mientras use el sitio desde el mismo navegador y tenga las cookies habilitadas.

\item Cuando un visitante quiera, se podrá abrir una cuenta (y quedará autenticado en ella), o autenticarse en una cuenta ya existente. En este caso, la funcionalidad quedará ligada a su cuenta.
  
\item Cada usuario podrá elegir cualquier ítem que se le presente, y realizar dos acciones fundamentales con él (comentarlo, y votarlo):
  \begin{itemize}
  \item Comentar un ítem quiere decir escribir un pequeño mensaje (menos de 256 caracteres) que quedará relacionado con el item.
  \item Votarlo significa darle un voto positivo o uno negativo. El resultado de las votaciones quedará relacionado con el ítem.
  \end{itemize}

\end{itemize}


%%----------------------------------------------------------------------------
\subsection{Alimentadores}
\label{practica-final-2020-05:alimentadores}

La práctica tendrá que funcionar con al menos dos alimentadores entre los que se describen a continuación. El número 0 (YouTube canal XML) será obligatorio para todos. Además, cada alumno tendrá que implementar al menos otro, según la primera letra de su primer apellido: alimentador 1 para las letras A-D, alimentador 2 para las letras E-L, alimentador 3 para las letras M-Q, alimentador 4 para las letras R-Z.

Alimentadores descritos:

\begin{itemize}
\item \textbf{Alimentador 0}. YouTube (canal XML). En este caso el alimentador será el canal de YouTube, y el item será un vídeo en particular. Los últimos vídeos de un canal están disponibles como documentos XML (RSS)\footnote{Ejemplo: Para el canal \verb|UC300utwSVAYOoRLEqmsprfg|, la url es:\\ \url{https://www.youtube.com/feeds/videos.xml?channel_id=UC300utwSVAYOoRLEqmsprfg}}, donde el identificador del canal se puede obtener del enlace que tenemos en el navegador cuando estamos viendo el canal. Funcionamiento:
  
  \begin{itemize}
  \item Alimentador: canal de Youtube.
  \item Ítem: vídeo de YouTube.
  \item Elemento HTML para elegir el alimentador: formulario que permita escribir el identificador del canal.
  \item Elemento HTML para actualizar el alimentador: botón que actualiza con los vídeos disponibles en el canal RSS.
  \item Datos mostrados para el alimentador cuando se muestra resumido: nombre (título) del canal, enlace del canal, total de items disponibles para este alimentador, puntuación (total de votos positivos menos votos negativos para todos sus items).
  \item Datos mostrados para el alimentador cuando se muestra con detalle: nombre (título) del canal, enlace del canal, y lista de vídeos (con información detallada).
  \item Datos mostrados del ítem (cuando se muestra resumido): título del vídeo, enlace del vídeo
  \item Datos mostrados del ítem (cuando se muestra con detalle): título del vídeo, enlace del vídeo, descripción del vídeo, vídeo empotrado, nombre del canal, enlace del canal.
  \end{itemize}

\item \textbf{Alimentador 1}. Reddit (Subreddit). En este caso, el alimentador será un Subreddit (una sección de Reddit, como por ejemplo \verb|r/memes|), y el item una noticia en el Subreddit. Las últmas noticias de un Subreddit están disponibles como documento XML (RSS)\footnote{Ejemplo: Para el Subreddit \verb|memes|, la url es:\\ \url{https://www.reddit.com/r/memes.rss}}. Más información en el wiki de Reddit\footnote{What features does reddit have?: \\ \url{https://www.reddit.com/wiki/rss}}. Funcionamiento:

  \begin{itemize}
  \item Alimentador: Subreddit (sección) de Reddit.
  \item Ítem: noticia del Subreddit.
  \item Elemento HTML para elegir el alimentador: formulario que permita escribir el nombre del Subreddit.
  \item Elemento HTML para actualizar el alimentador: botón que actualiza con las noticias disponibles en el canal RSS.
  \item Datos mostrados para el alimentador cuando se muestra resumido: nombre (título) del Subreddit, enlace del Subreddit, total de items disponibles para este alimentador, puntuación (total de votos positivos menos votos negativos para todos sus items).
  \item Datos mostrados para el alimentador cuando se muestra con detalle: nombre (título) del Subreddit, enlace del Subreddit, y lista de noticias (con información detallada).
  \item Datos mostrados del ítem (cuando se muestra resumido): título de la noticia, enlace de la noticia.
  \item Datos mostrados del ítem (cuando se muestra con detalle): título de la noticia, enlace de la noticia, descripción de la noticia, nombre del Subreddit, enlace del Subreddit.
  \end{itemize}

\item \textbf{Alimentador 2}. Last.fm (artista). En este caso, el alimentador será un artista de Last.fm (como por ejemplo \verb|Cher|), y el item un álbum. Los álbumes de un artista están están disponibles como documento XML o JSON\footnote{Ejemplo: Para \verb|Cher|, la url es:\\ \url{http://ws.audioscrobbler.com/2.0/?method=artist.gettopalbums&artist=cher&api_key=YOUR_API_KEY}}. Más información en la documentación ``Last.fm Web Services\footnote{Last.fm Web Services, artist.getTopAlbums: \\ \url{https://www.last.fm/api/show/artist.getTopAlbums}}. Funcionamiento:

  \begin{itemize}
  \item Alimentador: Artista en Last.fm.
  \item Ítem: álbum de un artista en Last.fm.
  \item Elemento HTML para elegir el alimentador: formulario que permita escribir el nombre del artista.
  \item Elemento HTML para actualizar el alimentador: botón que actualiza con los álbumes disponibles para el artista.
  \item Datos mostrados para el alimentador cuando se muestra resumido: nombre del artista, enlace al artista, total de items disponibles para este alimentador, puntuación (total de votos positivos menos votos negativos para todos sus items).
  \item Datos mostrados para el alimentador cuando se muestra con detalle: nombre del artista, enlace del artista, y lista de álbumes (con información detallada).
  \item Datos mostrados del ítem (cuando se muestra resumido): título del álbum, enlace del álbum.
  \item Datos mostrados del ítem (cuando se muestra con detalle): título del álbum, enlace del álbum, portada del álbum, nombre del artista, enlace del artista.
  \end{itemize}


\item \textbf{Alimentador 3}. Flickr (etiqueta). En este caso, el alimentador será una eitqueta (tag) de Flickr (como por ejemplo ``Fuenlabrada''), y el item una foto con esa etiqueta. Las fotos que tienen una etiqueta están disponibles como documento XML\footnote{Ejemplo: Para la etiqueta ``Fuenlabrada'' la url es: \\ \url{https://www.flickr.com/services/feeds/photos_public.gne?tags=fuenlabrada}}. Más información en la página Public Feed de Flickr\footnote{Public Feed de Flickr: \\ \url{https://www.flickr.com/services/feeds/docs/photos_public/}}. Funcionamiento:

  \begin{itemize}
  \item Alimentador: Etiqueta de Flickr.
  \item Ítem: foto de Flickr.
  \item Elemento HTML para elegir el alimentador: formulario que permita escribir la etiqueta.
  \item Elemento HTML para actualizar el alimentador: botón que actualiza con las fotos de la etiqueta.
  \item Datos mostrados para el alimentador cuando se muestra resumido: etiqueta, enlace a las fotos con esa etiqueta en Flickr\footnote{Por ejemplo, para la etiqueta ``Fuenlabrada'', el enlace sería: \\\url{https://www.flickr.com/search/?tags=fuenlabrada}}, total de items disponibles para este alimentador, puntuación (total de votos positivos menos votos negativos para todos sus items).
  \item Datos mostrados para el alimentador cuando se muestra con detalle: etiqueta, enlace a la página de la etiqueta en Flickr, y lista de fotos para esa etiqueta (con información detallada).
  \item Datos mostrados del ítem (cuando se muestra resumido): título de la foto, enlace a la página de la foto en Flickr.
  \item Datos mostrados del ítem (cuando se muestra con detalle): título de la foto, enlace a la página de la foto en Flickr, foto, etiqueta, enlace a la página de la etiqueta en Flickr.
  \end{itemize}

\item \textbf{Alimentador 4}. Wikipedia (historia de artículos). En este caso, el alimentador será la historia de un artículo de Wikipedia (como por ejemplo ``Madrid''), y el item la descripción de un cambio en esa historia. La historia de un artículo está disponible como documento XML\footnote{Ejemplo: Para la página ``Fuenlabrada'' la url es: \\ \url{https://en.wikipedia.org/w/index.php?title=Fuenlabrada&action=history&feed=rss}}. Más información en la página Wikipedia Syndication\footnote{Wikipedia Syndication: \\ \url{https://en.wikipedia.org/wiki/Wikipedia:Syndication}} (sección ``RSS Feeds''). Funcionamiento:

  \begin{itemize}
  \item Alimentador: historia de un artículo de Wiipedia.
  \item Ítem: cambio en la historia de un artículo.
  \item Elemento HTML para elegir el alimentador: formulario que permita escribir el nombre del artículo.
  \item Elemento HTML para actualizar el alimentador: botón que actualiza con la historia de un artículo.
  \item Datos mostrados para el alimentador cuando se muestra resumido: nombre del artículo, enlace al articulo en la Wikipedia, total de items disponibles para este alimentador, puntuación (total de votos positivos menos votos negativos para todos sus items).
  \item Datos mostrados para el alimentador cuando se muestra con detalle: nombre del artículo, enlace al articulo en la Wikipedia, y lista de cambios para ese articulo (con información detallada).
  \item Datos mostrados del ítem (cuando se muestra resumido): título del cambio, enlace al cambio.
  \item Datos mostrados del ítem (cuando se muestra con detalle): título del cambio, enlace al cambio, autor del cambio, fecha del cambio, nombre del artículo, enlace al articulo en la Wikipedia.
  \end{itemize}

\end{itemize}

Otros alimentadores, entre ellos algunos sugeridos por alumnos de la asignatura, por si te interesa implementarlos:

\begin{itemize}
\item TodoLiteratura (sección). En este caso, el alimentador será una sección de TodoLiteratura (como por ejemplo ``Actualidad''), y el item un artíclo de la sección. Las noticias de una sección están disponibles como documento XML\footnote{Ejemplo: Para la sección ``Actualidad'' se usa el número ``127'', y la url es: \\ \url{https://www.todoliteratura.es/rss/seccion/127/}}. Más información en la página de canales RSS de TodoLiteratura\footnote{Página de canales RSS de Todo Literatura: \\ \url{https://www.todoliteratura.es/rss/}}. Funcionamiento:

  \begin{itemize}
  \item Alimentador: Sección en TodoLiteratura.
  \item Ítem: noticia en una sección de TodoLiteratura.
  \item Elemento HTML para elegir el alimentador: formulario que permita escribir el identificador de la seccion (número de la sección). Alternativamente, se puede usar un menú que de cómo opción varias de las secciones.
  \item Elemento HTML para actualizar el alimentador: botón que actualiza con las noticias disponibles en una sección.
  \item Datos mostrados para el alimentador cuando se muestra resumido: título de la sección, enlace a la sección, total de items disponibles para este alimentador, puntuación (total de votos positivos menos votos negativos para todos sus items).
  \item Datos mostrados para el alimentador cuando se muestra con detalle: título de la noticia, enlace a la sección, descripción de la sección, y lista de noticias (con información detallada).
  \item Datos mostrados del ítem (cuando se muestra resumido): título de la noticia, enlace a la noticia.
  \item Datos mostrados del ítem (cuando se muestra con detalle): título de la noticia, enlace a la noticia, descripción de la noticia, título de la sección, enlace a la sección.
  \end{itemize}

\item TuCanaldeSalud (sección). En este caso, el alimentador será una sección de TuCanaldeSalud (como por ejemplo ``Tecnología''), y el item un artíclo de la sección. Las noticias de una sección están disponibles como documento XML\footnote{Ejemplo: Para la sección ``Tecnología'' se usa el número ``70038'', y la url es: \\ \url{https://www.tucanaldesalud.es/idcsalud-client/cm/tucanaldesalud/rss?locale=es_ES&rssContent=70038}}. Más información en la página de canales RSS de TuCanaldeSalud\footnote{Página de canales RSS de TuCanaldeSalud: \\ \url{https://www.tucanaldesalud.es/es/feed-rss}}. Funcionamiento:

  \begin{itemize}
  \item Alimentador: Sección en TuCanaldeSalud.
  \item Ítem: noticia en una sección de TuCanaldeSalud.
  \item Elemento HTML para elegir el alimentador: formulario que permita escribir el identificador de la seccion (número de la sección). Alternativamente, se puede usar un menú que de cómo opción varias de las secciones.
  \item Elemento HTML para actualizar el alimentador: botón que actualiza con las noticias disponibles en una sección.
  \item Datos mostrados para el alimentador cuando se muestra resumido: título de la sección, enlace a la sección, total de items disponibles para este alimentador, puntuación (total de votos positivos menos votos negativos para todos sus items).
  \item Datos mostrados para el alimentador cuando se muestra con detalle: título de la noticia, enlace a la sección, descripción de la sección, y lista de noticias (con información detallada).
  \item Datos mostrados del ítem (cuando se muestra resumido): título de la noticia, enlace a la noticia.
  \item Datos mostrados del ítem (cuando se muestra con detalle): título de la noticia, enlace a la noticia, descripción de la noticia, título de la sección, enlace a la sección.
  \end{itemize} 
\end{itemize}

Además de las anteriores, puedes proponer otros alimentadores. Los requisitos fundamentales son que sean accesibles públicamente (el acceso mediante un token de aplicación se considera público), y que proporcionen datos en formato XML o JSON. Si hay algún alimentador que querrías utilizar, coméntalo con los profesores para que te indiquen si es un alimentador válido. En caso de ser aceptado como válido, estos alimentadores serán puntuados positivamente, teniendo en cuenta la iniciativa del alumno que los propuso.

Si quieres buscar servicios que ofrezcan APIs que podrían ser alimentadores, puedes buscarlos en Internet. Una lista por la que puedes comenzar es la que mantiene ProgrammableWeb\footnote{Programmable Web API Directory: \\\url{https://www.programmableweb.com/apis/directory}}.

%%----------------------------------------------------------------------------
\subsection{Funcionalidad mínima}

La aplicación servirá las siguientes páginas:

\begin{itemize}
  \item Página principal de la aplicación:
  
    \begin{enumerate}
    \item Listado con los 10 items (formato resumido) que han conseguido más puntuación (votos positivos menos votos negativos) en el sitio. Para cada uno se mostrarán sus votos positivos y negativos, y un enlace a la página del item (ver a continuación).
    \item Formulario para elegir alimentador, para cada uno de los sistemas de alimentación (por ejemplo, canales de YouTube) disponibles. Tras elegir un alimentador vía el formulario, se recibirá la página del alimentador elegido (ver a continuación), con información actualizada, y se almacenarán sus datos en la base de datos (todos los recibidos, si es la primera vez que se le ha elegido, o lo que no estuvieran ya en la base de datos, si ya se hubiera elegido anteriormente).
    \item Listado de alimentadores elegidos en el pasado (formato resumido), por cualquier usuario. Cada alimentador aparecerá con un botón para poder elegirlo (si se elige de esta forma, la aplicación se comportará igual que si se hubiera elegido vía el formulario), y otro para eliminarlo (si se pulsa, el alimentador dejará de salir en este listado en el futuro). Cualquier visitante o usuario podrá eliminar un canal de este listado, pero eso no supondrá que sus datos desaparezcan de la base de datos, y en cualquier caso el alimentador seguirá saliendo en la página de alimentadores.
    \end{enumerate}

    Si el visitante está además autenticado como usuario, se mostrará también:

    \begin{itemize}
    \item Listado con los 5 items (formato resumido) que más ha votado el usuario (votos positivos menos votos negativos).
    \item Para cada item que aparezca en la página se mostrarán también dos botones para votar (positivo, negativo), resaltando de alguna forma que el valor que se haya votado, si se hubiera votado ya ese item, y un enlace a la página del item (ver a continuación).
    \end{itemize}

  \item Página del ítem (para cada ítem):

    \begin{itemize}
    \item Datos del ítem (formato detallado).
    \item Datos del alimentador al que pertenece el ítem (formato resumido), incluyendo un enlace a la página del alimentador.
    \item Comentarios que haya recibido el ítem. Para cada comentario se mostrará el texto del comentario, el identificador de quien lo puso, y la fecha en que se puso.
    \end{itemize}

    Si el visitante está además autenticado como usuario, se mostrará también:

    \begin{itemize}
    \item Dos botones para votar (positivo, negativo), resaltando de alguna forma que el valor que se haya votado, si se hubiera votado ya ese item.
    \item Formulario para poner un comentario. Tras poner el comentario, se volverá a ver la misma página del ítem.
    \end{itemize}

  \item Página de alimentadores:

    \begin{itemize}
    \item Listado de todos los alimentadores de los que se ha podido descargar datos alguna vez (formato resumido). Esto es, todos los que se han ``seleccionado'' alguna vez, por cualquier visitante, aunque no salgan en la página principal.
    \end{itemize}
    
  \item Página de alimentador (para cada alimentador):

    \begin{itemize}
    \item Datos del alimentador (formato detallado)
    \item Botón para poder elegir o dejar de tener elegido el alimentador. Si se pulsa, y no estaba elegido, el alimentador pasará a estar elegido, con los mismos efectos que si se hubiera elegido en el formulario de la página principal. Si se pulsa, y estaba elegido, pasa a dejar de estar elegido, con el mismo efecto que se hubiera pulsado el botón de ``eliminar'' del listado de alimentadores elegidos de la pagina principal. el botón tendrá que indicar de alguna manera (por ejemplo, con dos textos distintos, o con colores distintos) si el alimentador está o no elegido, antes de pulsarlo. En ningún caso si un alimentador pasa a dejar de estar elegido, se eliminarán sus datos de la base de datos: sólo dejará de salir en el listado de elegidos.
    \item Lista de items del alimentador (formato resumido).
    \item Para cada ítem, enlace a la página del ítem.
    \end{itemize}

  \item Página de usuario (para cada usuario ``con cuenta''):

    \begin{itemize}
    \item Datos públicos del usuario (identificador, foto)
    \item Lista de alimentaros elegidos (formato resumido)
    \item Lista de items votados (formato resumido)
    \item Lista de items comentados (formato resumido)
    \end{itemize}

    Si el visitante está autenticado, cuando acceda a su propia página, se mostrará también:

    \begin{itemize}
    \item Formulario para cambiar la foto
    \item Formulario para cambiar de estilo. Se ofrecerán al menos dos estilos: ``ligero'' y ``oscuro''.
    \item Formulario para cambiar el tamaño de la letra. Se ofrecerán al menos tres tamaños: ``pequeña'', ``normal'' y ``grande''.
    \end{itemize}
    
  \item Página de usuarios:

    \begin{itemize}
    \item Listado de todos los usuarios ``con cuenta''. Para cada usuario, aparecerá su identificador, su foto, el número de items votados, el número de canales elegidos, y un enlace a su página.
    \end{itemize}

  La página principal se ofrecerá también como un documento XML y como un documento JSON, que incluiría la misma información (los mismos listados de items y alimentadores). Este documento se ofrecerá cuando se pida la página principal, concatenando al final \verb|?format=xml| o \verb|?format=json|.

  La página principal en formato HTML incluirá un enlace a la página principal en formato XML (``Descarga como fichero XML'') y JSON (``Descarga como fichero JSON'').  
 
  \item Página de información: Página con información en HTML indicando la autoría de la práctica, explicando su funcionamiento y una brevísima documentación.

\end{itemize}

La aplicación se encargará de controlar que no haya más de un voto (positivo o negativo) por usuario para cada ítem. Por lo tanto, si un usuario ya ha votado un ítem, y vuele a votarlo, se ignorará su voto (si es igual que el que está almacenado) o se anotará el nuevo (si es distinto). Por ejemplo, si había votado un ítem con positivo, y ahora vuelve a votarlo con positivo, se ignorará el segundo voto. Si vuelve a votarlo, pero ahora con negativo, se cambiará el voto a negativo.

En todos los casos en que se vote, tras votar se volverá a ver la misma página en que se estaba, ahora con el voto contabilizado.

Todas las páginas un menú desde el que se podrá acceder a la página principal (con el texto ``Inicio''), a la de alimentadores (con el texto ``Alimentadores''), a la de usuarios (con el texto ``Usuarios'') y a la de información (con el texto ``Información''), salvo que ya estés en esa página, en cuyo caso no saldrá el elemento de menú correspondiente.

Además, la práctica incluirá tests, que se ejecutarán con \verb|python3 manage.py test|, y que incluirán al menos un test de API HTTP para cada recurso que sirva la aplicación, y para cada método (GET, POST) que admita cada recurso. Además, al menos la mitad de los test incluirán comprobar algo distinto del código HTTP retornado por la petición.

%%----------------------------------------------------------------------------
\subsection{Despliegue}

La práctica deberá estar desplegada en algún sitio de Internet, de forma que pueda accederse a ella. Deberá mantenerse desplegada y activa al menos desde el día de entrega de la práctica, hasta el día del cierre de actas.

Para el despliegue, se puede utilizar Python Anywhere\footnote{Python Anywhere: \url{https://pythonanywhere.com}}, que proporciona un plan gratuito que incluye suficientes recursos como para poder desplegar la práctica.

Si el alumno así lo desea, puede considerarse desplegar en un ordenador dedicado (por ejemplo, una Raspberry Pi accesible directamente desde Internet, alojada en su hogar), o en servicios como Google Computing Engine\footnote{GCP Engine Free: \url{https://cloud.google.com/free/}}. En general, dado que este tipo de despliegues no podrá contar con una ayuda detallada por los profesores, estará algo más valorado.

%%----------------------------------------------------------------------------
\subsection{Funcionalidad optativa}

De forma optativa, se podrá incluir cualquier funcionalidad relevante en el contexto de la asignatura. Se valorarán especialmente las funcionalidades que impliquen el uso de técnicas nuevas, o de aspectos de Django no utilizados en los ejercicios previos, y que tengan sentido en el contexto de esta práctica y de la asignatura.

En el formulario de entrega se pide que se justifique por qué se considera funcionalidad optativa lo que habéis implementado. Sólo a modo de sugerencia, se incluyen algunas posibles funcionalidades optativas:

\begin{itemize}
  \item Inclusión de un \emph{favicon} del sitio
  
  \item Visualización de cualquier página en formato JSON y/o XML, de forma similar a como se ha indicado para la página principal.

  \item Generación de un canal RSS, XML libre y/o JSON para los comentarios puestos en el sitio.

  \item Incorporación de datos de otros alimentadores además de los obligatorios. Se valorará especialmente la búsqueda de otros alimentadores no descritos en este enunciado, la implementación de alimentadores no basados en RSS (o derivados), y la implementación de alimentadores que requieran de token de autenticación (en este caso, atención a no subir el token de autenticación a GitLab).
 
  \item Atención al idioma indicado por el navegador. El idioma de la interfaz de usuario de la aplicación tendrá en cuenta lo que especifique el navegador.

  \item Utilización de Bootstrap\footnote{Bootstrap: \url{https://getbootstrap.com/}} para la maquetación del sitio web.

  \item Inclusión de imágenes (no solo texto) en los comentarios. Esto puede hacerse de dos formas: quien suba un comentario, además de rellenar una caja de texto con el comentario, puede indicar también la url de una imagen, que se mostrará junto al comentario, o bien subiendo una imgen a la aplicación, que se mostrará junto al comentario (se valorará más la segunda opción, y se pueden implementar las dos).
    
  \item Mejora de los tests de la práctica, incluyendo test de condiciones de error, test de escenarios con más de una invocación de recurso, tests de API Python, etc.
\end{itemize}

%%----------------------------------------------------------------------------
\subsection{Entrega de la práctica}

\begin{itemize}
  \item \textbf{Fecha límite de entrega de la prueba teórica:} viernes, 12 de junio de 2020 a las 15:00 (hora española peninsular)
       %{\bf Convocatoria de junio:} miércoles, 24 de junio de 2015 a las 23:59 (hora peninsular española).

  \item \textbf{Fecha límite de entrega de la práctica:} domingo, 14 de junio de 2020 a las 23:59 (hora española peninsular)

  \item \textbf{Notificación de alumnos que tendrán que realizar entrevista:} martes, 16 de junio, en el aula virtual.
%{\bf Convocatoria de junio:} viernes, 26 de junio, en la plataforma Moodle.

  \item \textbf{Realización de entrevistas:} miércoles, 17 de junio, en la aplicación Teams. Si es necesario, se realizarán también los días 18 y 19.
    
  \item \textbf{Fecha de publicación de notas:} viernes, 19 de junio, en el aula virtual.
%{\bf Convocatoria de junio:} viernes, 26 de junio, en la plataforma Moodle.

  \item \textbf{Fecha de revisión:} lunes, 22 de junio, a las 12:00, en la aplicación Teams.
%{\bf Convocatoria de junio:} martes, 30 de junio a las 13:30. Se requerirá a algunos alumnos que asistan a la revisión {\bf en persona}; se informará de ello en el mensaje de publicación de notas.
\end{itemize}

La entrega de la práctica consiste en:

\begin{enumerate}

  \item {\bf Rellenar un formulario} enlazado en el sitio de la asignatura en el aula virtual.
  
  \item {\bf Subir tu práctica a un repositorio en el GitLab de la Escuela}. El repositorio contendrá todos los ficheros necesarios para que funcione la aplicación (ver detalle más abajo). Es muy importante que el alumno haya realizado una derivación (fork) del repositorio que se indica a continuación, porque si no, la práctica no podrá ser identificada: 

\url{https://gitlab.etsit.urjc.es/cursosweb/practicas/server/final-miscosas/}

Recordad que es importante ir haciendo commits de vez en cuando y que sólo al hacer push estos commits son públicos. Antes de entregar la práctica, haced un push. Y cuando la entreguéis y sepáis el nombre del repositorio, podéis cambiar el nombre del repositorio desde el interfaz web de GitLab. 

Se recomeinda mantener el repositorio como privado, hasta el momento en que se entregue la práctica.

 \item {\bf Entregar un vídeo de demostración de la parte obligatoria, y otro vídeo de demostración de la parte opcional}, si se han realizado opciones avanzadas. Los vídeos serán de una {\bf duración máxima de 3 minutos} (cada uno), y consistirán en una captura de pantalla de un navegador web utilizando la aplicación, y mostrando lo mejor posible la funcionalidad correspondiente (básica u opcional). Siempre que sea posible, el alumno comentará en el audio del vídeo lo que vaya ocurriendo en la captura. Los vídeos se colocarán en algún servicio de subida de vídeos en Internet (por ejemplo, Vimeo, Twitch, o YouTube). Los vídeos de más de tres minutos tendrán penalización.

Hay muchas herramientas que permiten realizar la captura de pantalla. Por ejemplo, en GNU/Linux puede usarse Gtk-RecordMyDesktop o Istanbul (ambas disponibles en Ubuntu). OBS Studio\footnote{OBS Studio: \url{https://obsproject.com/}} está disponible para varias plataformas (Linux, Windows, MacOS). Es importante que la captura sea realizada de forma que se distinga razonablemente lo que se grabe en el vídeo.

En caso de que convenga editar el vídeo resultante (por ejemplo, para eliminar tiempos de espera) puede usarse un editor de vídeo, pero siempre deberá ser indicado que se ha hecho tal cosa con un comentario en el audio, o un texto en el vídeo. Hay muchas herramientas que permiten realizar esta edición. Por ejemplo, en GNU/Linux puede usarse OpenShot o PiTiVi.

\end{enumerate}

Sobre la entrega del repositorio:
\begin{itemize}
  \item Se han de entregar los siguientes ficheros:

\begin{itemize}
  \item El repositorio en la instancia GitLab de la ETSIT deberá contener un proyecto Django completo y listo para funcionar en el entorno del laboratorio, incluyendo la base de datos. Deberá poder ejecutarse directamente con \verb|python3 manage.py runserver| desde un entorno virtual en el que esté instalado Django~3.0.3. Tmbién ejecutará los tests con \verb|python3 manage.py test|, desde el mismo entorno virtual.

  \item La base de datos habrá de tener datos suficientes como para poder probarlo. Estos datos incluirán al menos dos usuarios con sus datos correspondientes, con al menos cinco alimentadores elegidos en total, cinco comentarios puestos en total, y al menos 10 items votados por cada usuario.

  \item Un fichero \verb|requirements.txt|, con un nombre de paquete Python por línea, para indicar Cualquier biblioteca Python que pueda hacer falta para que la aplicación funcione, si es que fuera el caso. Este fichero no ha de incluir Django, dado que ya se supone que hace falta. Si es posible, se recomienda escribir este fichero en el formato que entiende \verb|pip install -r requirements.txt|

  \item Cualquier fichero auxiliar que pueda hacer falta para que funcione la práctica, si es que fuera el caso.
\end{itemize}

  \item Se incluirán en el fichero README.md los siguientes datos (la mayoría de estos datos se piden también en el formulario que se ha de rellenar para entregar la práctica: se recomienda hacer un copia y pega de estos datos en el formulario):

\begin{itemize}
  \item Nombre y titulación.
  \item Nombre de su cuenta en el laboratorio del alumno.
  \item URL del vídeo demostración de la funcionalidad básica
  \item URL del vídeo demostración de la funcionalidad optativa, si se ha realizado funcionalidad optativa
  \item URL de la aplicación desplegada
  \item Cuenta (login) y contraseña de los usuarios que están dados de alta en la aplicación.
  \item Resumen de las peculiaridades que se quieran mencionar sobre lo implementado en la parte obligatoria.
  \item Lista de funcionalidades opcionales que se hayan implementado, y breve descripción de cada una.
\end{itemize}

Estos datos se escribirán siguiendo estrictamente el siguiente formato:

\begin{verbatim}
# Entrega practica

## Datos

* Nombre:
* Titulación:
* Despliegue (url):
* Video básico (url):
* Video parte opcional (url):
* Despliegue (url):
*

## Cuenta Admin Site

* usuario/contraseña

## Cuentas usuarios

* usuario/contraseña
* usuario/contraseña
* ...

## Resumen parte obligatoria

## Lista partes opcionales

* Nombre parte:
* Nombre parte:
* ...
\end{verbatim}

Asegúrate de que las URLs incluidas en este fichero están adecuadamente escritas en Markdown, de forma que la versión HTML que genera GitLab los incluya como enlaces ``pinchables''.
\end{itemize}


%%----------------------------------------------------------------------------
\subsection{Notas y comentarios}

La práctica deberá funcionar en el entorno GNU/Linux (Ubuntu) del laboratorio de la asignatura con la versión de Django que se ha usado en prácticas.

La práctica deberá funcionar desde el navegador Firefox disponible en el laboratorio de la asignatura.

Los canales (feeds) RSS que produce la aplicación web realizada en la práctica deberán funcionar al menos con el navegador Firefox (considerándolos como canales RSS) disponibles en el laboratorio. Los documentos XML deberán ser correctos desde el punto de vista de la sintaxis XML, y por lo tanto reconocibles por un reconocedor XML, como por ejemplo el del módulo xml.sax de Python. Los documentos JSON generados deberán ser correctos desde el punto de vista de la sintaxis JSON, y por lo tanto reconocibles por un reconocedor JSON, como por ejemplo el del módulo json de Python

%%----------------------------------------------------------------------------
\subsection{Preguntas y respuestas}

A continuación, algunas preguntas relacionadas con el enunciado de esta práctica, junto con sus respuestas:

\begin{itemize}

\item En el formulario para elegir alimentador, ¿qué hay que introducir? (el nombre de los alimentadores que ya hay, dar opciones de los que ya hay, directamenet el identificador de alimentador...).

  Habrá un formulario por cada tipo de alimentador. Por ejemplo, uno para canales de YouTube, otro para eitquetas de Flickr, etc. Cada uno de ellos será un formulario en el que se podrá poner lo que haga falta para elegir un alimentador para ese tipo de alimentador. Por ejemplo, en el caso de Youtube, el formulario tendrá una caja de texto para poder poner el id del canal, y un botón para enviarlo. Lo normal, es que todos los formularios para los tipos de alimentadores que hayas implementado estén juntos.

  Alternativamente, y esto es opcional, se puede tener un único formulario para todos los tipos de alimentador. En ese caso, tendrás que tener algo parecido a un menú desplegable para que puedas elegir qué tipo de alimentador vas a indicar, algún elemento para poner el idenficador (id, etiqueta, nombre de sección... lo que sea), y un botón para enviar.

  Para algunos alimentadores (por ejemplo, el de TuCanaldeSalud), puede tener sentido que el formulario incluya un menú par elegir el alimentador, pero este no es el caso de ninguno de los alimentadores obligatorios. Puede ocurrir esto cuando el número de alimentadores a elegir sea pequeño.
  
\item ¿Es necesario utilizar los mecanismos provistos por Django para el control de sesiones y autenticación?

  En principio, esa es la solución recomendada. El principal problema suele ser asegurarse de que cuaquier mecanismo alternativo funciona al menos tan bien como el de Django, lo que no es en general trivial. De todas formas, salvo muy buenos motivos, la aplicación es una aplicación Django, y por lo tanto cuantas más facilidades de Django se usen (bien usadas), mejor.

\item Los archivos CSS que pueden modificar los usuarios, ¿dónde y cómo debemos guardarlos?

  La forma recomendada de hacerlo es mediante plantillas:

  \begin{itemize}
  \item En el directorio de plantillas incluirías una para la hoja CSS del sitio. Esa plantilla tendría como variables de plantilla los valores que quieras que los usuarios puedan cambiar (color de tipo de letra, tamaño de tipo de letra, etc.).
  \item Además, para cada usuario, tendrás una tabla en la base de datos donde se almacenarán los valores para ese usuario (normalmente, una fila de la tabla por usuario).
  \item Tendrás una vista en views.py que se encargará de generar la hoja CSS a partir de la plantilla. Esa vista es la que comprobará si la petición que está atendiendo corresponde a un usuario (en cuyo caso tendrá que obtener los valores para ese usuario de la tabla anterior), o no (en cuyo caso usará valores por defecto). Con los valores que obtenga, generará la hoja CSS a partir de la plantilla anterior.
  \item Por último, en urls.py tendrás una línea para indicar que si te piden el recurso que sirve la hoja de estilo, llamas a la vista anterior.
  \end{itemize}

\item ¿Dónde puedo realizar el despliegue de la aplicación?

  El despliegue puede realizarse en cualquier ordenador que esté conectado permanentemente a Internet durante el periodo de correción, en una dirección accesible desde cualquier navegador conectado a su vez a Internet. Esto puede ser por ejemplo un ordenador personal en un domicilio con acceso permanente a Intener, adecaudamente configurado (puede ser una Raspberry Pi o similar, si se busca una solución simple y de bajo coste). También puede ser un servicio en Internet, por ejemplo uno gratuito como los que ofrecen Google (instrucciones\footnote{GCP Quickstart Using a Linux VM:\\ \url{https://cloud.google.com/compute/docs/quickstart-linux}}, precios\footnote{Google Compute Engine Pricing:\\ \url{https://cloud.google.com/compute/pricing}}), o PythonAnywhere (instrucciones\footnote{Capítulo ``Deploy!'' de Django Girls Tutorial:\\ \url{https://tutorial.djangogirls.org/en/deploy/}}, precios\footnote{PythonAnywhere Plans and Pricing:\\ \url{https://www.pythonanywhere.com/pricing/}}). Los profesores podremos ayudar de forma más detallada con PythonAnywhere.

\end{itemize}
